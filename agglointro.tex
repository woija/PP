
I dette afsnit vil vi gennemgå den økonomiske teori bag agglomeration og baggrunden for, hvorfor agglomeration påvirker produktivitet. Introduktionen til begrebet er primært baseret på bagrund af bogen \emph{"Econonmics of Agglomeration"} (2013) af Fujita og Thisse.

\subsection{Definitioner og introduktion}
Agglomerationsøkonomi betegner den eskternalitet, der opstår, når økonomiske personer eller virksomheder geografisk placeres tættere på hinanden. 

Agglomeration er indenfor den økonomiske litteratur et relativt nyt aspekt. Især inddrages agglomeration nu i højere grad i estimeringer af eksterne økonomiske effekter af transportomkostninger, hvor den positive eksternalitet ved at samlokalisere agenter indtil for nyligt ikke har været tilstrækkeligt inddraget i cost-benefit analyser.
%
Motivationen for at inddrage agglomeration i sådanne analyser er, at investeringer i infrastruktur sænker den effektive distance mellem økonomiske agenter, hvilket både har interne og eksterne effekter. De interne effekter er fx mindre rejsetid, færre transportomkostninger, mens de eksterne effekter ved agglomeration er gennem videnskudveksling, større innovation, større specialisering. Relevansen for disse eksterne effekter i en vores sammenhæng er, at de medfører en forøgelse af produktiviteten.

Definitionen på agglomerationen er derfor den paraply, der overlapper alle effekter ved geografisk at samle økonmiske agenter (CITATION). Disse effekter kan både være positive -- som eksempelvis større innovation og vidensspillover -- mens øget trængsel ved for mange personer på samme sted indgår negativt.



