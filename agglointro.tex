
I dette afsnit vil vi gennemgå den økonomiske teori bag agglomeration og baggrunden for, hvorfor agglomeration påvirker produktivitet. Introduktionen til begrebet er primært baseret på bagrund af bogen \emph{"Econonmics of Agglomeration"} (2013) af Fujita og Thisse.

\subsection{Definitioner og introduktion}
Agglomerationsøkonomi betegner en positivt eskternalitet som opstår, når økonomiske personer eller virksomheder drager nytte af være fysisk tæt på hinanden. 

Agglomeration er indenfor den økonomiske litteratur et relativt nyt aspekt. Især har det været benyttet til at approksimere de brede eller eksterne økonomisek effekter af transportomkostninger, der tidligere ikke har været tilstrækkeligt inddraget i typiske cost-benefit analyser.

Motivationen for at inddrage agglomeration i sådanne analyser er, at investeringer i infrastruktur sænker den effektive distance mellem økonomiske agenter, hvilket både har interne og eksterne effekter. De interne effekter er fx mindre rejsetid, færre transportomkostninger, mens de eksterne effekter ved agglomeration er gennem videnskudveksling, større innovation, større specialisering. Relevansen for disse eksterne effekter i en vores sammenhæng er, at de medfører en forøgelse af produktiviteten.

Definitionen på agglomeration er angivet

Intuitionen bag denne er lang


