
I dette afsnit vil vi gennemgå den økonomiske teori bag agglomeration og baggrunden for, hvorfor agglomeration påvirker produktivitet. Introduktionen til begrebet er primært baseret på bagrund af bogen \emph{"Econonmics of Agglomeration"} (2013) af Fujita og Thisse.

 og 

Agglomerationsøkonomi betegner en positivt eskternalitet som opstår, når økonomiske agenter (personer og virksomheder) drager nytte af være fysisk tæt på hinanden. Denne effekt Teorisk kan denne effekt opstå af flere forskellige årsager. 


