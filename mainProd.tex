%%LATEX header ********************************************
\documentclass[a4paper, 12pt]{article}
\usepackage[utf8]{inputenc}
\usepackage[T1]{fontenc}
%\usepackage[danish]{babel}
\usepackage{cite,refstyle,adjustbox,graphicx,amsmath,amssymb,subcaption,varioref,titlesec,comment,amsthm}
\usepackage{mathtools}
%\usepackage{epstopdf}
\usepackage{booktabs}
\usepackage{geometry}
\usepackage{setspace}
\usepackage{pgf,tikz}
\usepackage{fancyvrb}
%%************************************************************
%%Teorem miljøer 
\newtheorem{antag}{Antagelse}
\newtheorem{teo}{Teorem}
%%************************************************************
%%Andet setup
\onehalfspacing
%%************************************************************
%% Makro
\newcommand{\indep}{\perp\!\!\!\perp}
\newcommand{\saa}{\Leftrightarrow}
\newcommand{\pd}[2]{\frac{\partial #1}{\partial #2}}
\newcommand{\Lagr}{\mathcal{L}}
%%************************************************************
%% Om forfatteren
%%************************************************************
\author{Simon Harmat og Niels Bækgård}
\title{Øvelse i produktivitetspolitik}
\date{\today}
\begin{document}
<<<<<<< Updated upstream
=======
<<<<<<< HEAD
Hermed en analyse og vurdering af produktivitetsudfordringer særligt i Region sjælland
=======
>>>>>>> Stashed changes
\section{Indledning}
Vi er interesserede i at undersøge, om der kunne være uudtømte effekt ved fra urbanisering i region Hovedstaden set i forhold til region Sjælland. 
\subsection{Motivation}
Der er store forskelle i timeproduktiviteten mellem region Sjælland og region Hovedstaden. En del af disse forskelle skyldes ganske givet branchesammensætning, hvor mere produktive brancher fylder mere i region Hovedstaden end de gør i region Sjælland. Men det kan ikke forklare det hele.
<<<<<<< Updated upstream
=======
>>>>>>> origin/master
>>>>>>> Stashed changes

Hvis man i stedet sammenligner samme brancher og dermed ser på, hvad produktivitetsforskellen måtte være her, da vil det være muligt at udrede om der gives urbane produktivitetseffekter. Dette kalder vi for \emph{urban learning}.

\section{Estimation}
Vi ønsker at estimere følgende output dataset
\begin{equation}
	\ln Y_{it}^{pj} = \alpha^p_0 + \ln K_{it} + \ln L_{it} + ED^j_{t} + \omega^p_{t}
\end{equation}
hvor $i$ er virksomhedsindeks, $p$ kommuneindeks, $t$ angiver tidspunkt i år og $j$ angiver kommune. 

<<<<<<< Updated upstream
=======
<<<<<<< HEAD
haha 

Niels 
=======
>>>>>>> Stashed changes
$ED$ er en afstandsparameter, der måler, hvor meget arbejdskraft der der tilgængelig for den enkelte virksomhed, når der tages højde for den geografiske nærhed. Således er det i modellen muligt for en virksomhed i Lolland kommune at rekrutere arbejdskraft i Aalborg, men disse 

\begin{equation}
 	ED^j_t = \frac{L^j_t}{\sqrt{\frac{A_j}{\pi}}} + \sum_{k=1}^{k \neq j} \frac{L_p^t}{d_{kj}}
 \end{equation} 
 hvor $L$
<<<<<<< Updated upstream
=======
>>>>>>> origin/master
>>>>>>> Stashed changes


\bibliography{prodBib.bib}
\bibliographystyle{plain}
\end{document}