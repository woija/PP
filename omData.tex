\subsection{Input til produktionsfunktionen}
\textbf{Input til $l_t$ er Kvalitetskorrigeret arbejdskraft}
Når virksomhedernes anvendelse af arbejdskraft indgår i produktionen, er det i et produktivitetsøjemed vigtigt at tage højde for, at medarbejderens produkvitet afhænger af især uddannelse og erfaring.

Derfor har vi et kvalitetskorrigeret arbejdsindeks som arbejdsinput i produktionsfunktionen \cite[s. 5]{dors2010baggrund}. Denne tager højde for, hvor meget den enkelte virksomhed inddrager af arbejdskraft, så der tages højde for medarbejderens uddannelse og erfaring. Uddannelsesniveau og erfaring kombineres i Danmark Statistiks registre med indberettede regnskabsdata så der for den enkelte virksomhed kan skabes et kvalitetskorrigeret arbejdsinput.

Kvalitet følger DØRS definition og udregnes som \cite[s. 5]{dors2010baggrund}:
\begin{equation}
	\bar{L_{it}} = 
	L_{i0t} + \sum_{f=1}^{F-1} \frac{\bar{w}_{fgt}}{\bar{w}_{0gt}} 
	L_{ift}
\end{equation}
$f$ er her en gruppering af beskæftigelsen efter kvalitet. Fx er $f=0$ en ufaglært medarbejder med lav erfaring, $f=1$ er en ufaglært medarbejder med 5-10 års erfaring osv. CHECK!!!!!!
$L_{ift}$ er så antallet af årsværk i virksomheder med kvalitet $f$. Dermed er det muligt at korrigere for kvaliteten af arbejdskraften, så en times præsteret nu tæller forskelligt alt efter produktivteten.

I produktionsfunktionen fra (\ref{eq:LPProdFunkt02}) indgår den kvalitetskorrigerede arbejdskraft som $l_{it} = \log(L_{it})$.  

\section{Beskrivelse af data}
Vi har benyttet Danmarks Statistiks virksomhedsregister som baggrund for estimeringen. Data tilgængeligt for os er i årene 2008 -- 2013, hvori vi har 45.751 observationer fordelt på ca. 17.000 cvr numre. Datasættet er ikke et fuldt panel. For en beskrivelse af agglomerationsvariablen, jf. kapital ???.

Virksomhederne er kun indenfor branche i private byerhverv, hvilket vil sige branchegrupperne Industri, Bygge og anlæg, Transport, Hoteller og restauranter, Handel, Information og Kommunikation, Videnservice og Operationel service.

Der er lavet visse restrektioner på registrene. Blandt andet er virksomheder med færre end fem ansatte. 

\paragraph{Kapitalapparat} 
Kapitalinput er virksomhedernes materielle og immatrielle aktiver fra regnskabsstatistikken, hvorefter den deflateres. Denne deflatering sker ved hjælp af deflateren for kapital fra Danmarks Statistiks nationalregnskab for de relevante år\footnote{Fordelt på branche??}.

\begin{itemize}
	\item Spørg Lill om metoden
\end{itemize}

\paragraph{Antagelser for konsistent estimering af ed}
%%Det jeg vil skrive nu
% - Bootstrapping
% - Bootstrapping