

Hermed en analyse og vurdering af produktivitetsudfordringer særligt i Region sjælland

\section{Indledning}
Vi er interesserede i at undersøge, om der kunne være uudtømte effekt ved fra urbanisering i region Hovedstaden set i forhold til region Sjælland. 
\subsection{Motivation}
Der er store forskelle i timeproduktiviteten mellem region Sjælland og region Hovedstaden. En del af disse forskelle skyldes ganske givet branchesammensætning, hvor mere produktive brancher fylder mere i region Hovedstaden end de gør i region Sjælland. Men det kan ikke forklare det hele.

Hvis man i stedet sammenligner samme brancher og dermed ser på, hvad produktivitetsforskellen måtte være her, da vil det være muligt at udrede om der gives urbane produktivitetseffekter. Dette kalder vi for \emph{urban learning}.

\section{Estimation}
Vi ønsker at estimere følgende output dataset
\begin{equation}
	\ln Y_{it}^{pj} = \alpha^p_0 + \ln K_{it} + \ln L_{it} + ED^j_{t} + \omega^p_{t}
\end{equation}
hvor $i$ er virksomhedsindeks, $p$ kommuneindeks, $t$ angiver tidspunkt i år og $j$ angiver kommune. 


$ED$ er en afstandsparameter, der måler, hvor meget arbejdskraft der der tilgængelig for den enkelte virksomhed, når der tages højde for den geografiske nærhed. Således er det i modellen muligt for en virksomhed i Lolland kommune at rekrutere arbejdskraft i Aalborg, men disse 

\begin{equation}
 	ED^j_t = \frac{L^j_t}{\sqrt{\frac{A_j}{\pi}}} + \sum_{k=1}^{k \neq j} \frac{L^k_t}{d_{kj}}
 \end{equation} 
 hvor $d_{kj}$ er afstanden mellem kommune $k$ og $j$. 

\paragraph{Kvalitetsjusteret arbejdskraft}




Hvis agglomerationen fremmes, altså at afstanden i økonomien mindskes, kan der samtidigt opstå problemer. Det første problem kunne fx. opstå hvis en by oplever stor tilflytning, og at der samtidigt ikke investeres propertionelt i infrastukturen, så risiker man at opleve trængsel i området. Trængsel koster tid, længere rejsetider betyder en øget omkostning for virksomheder og gør derfor produktionen dyrer. Omkostningen kan være direkte, såsom løn til chauffører, sælgere og kørende konsulenter.Trængsel kan også skabe et behov for mere kapital i form af flere transportmidler, fordi fremkommeligheden på tværs af en by bliver svær. Indirekte kan også tænke på, at pendlende medarbejder får en højere reservationsløn pga. rejsetiden, og der gør matching mellem virksomheder og medarbejdere svær \cite{sorensen2014infrastruktur}. Et andet problem kan være i forbindelse med den forurening som opstår ved urbanisering. OECD påpeger at væsentlige samfundsøkonomiskeomkostninger ved forurening. Så hvis ikke der foretages politisk initiativer til at mindske forureningen vil samfundet få øget omkostning eksempelvist gennem fald i produktiviteten pga. dårlig helbred, højere dødelighed, øget sundhedsinvestering og faldene udbytte ved landbrug \cite{klimonteconomic}.




@article{klimonteconomic,
  title={The Economic Consequences of Air Pollution},
  author={Klimont, Zbigniew and IIASA, Fabian Wagner}
}

Når infrastrukturen forbedres, eller ny infrastruktur anlægges, er den primære effekt oftest kortere rejsetid. Det betyder, at virksomhederne sparer løn- og kapitalomkostninger til vare- transport og ved forretningsrejser. Det gør produktionen billigere.
Ud over selve rejsetiden har det også en værdi at vide på forhånd, hvor lang rejsetiden vil være. Det gælder fx, hvis virksomhedens logistiske set-up er baseret på leveringer på be- stemte tidspunkter. Når der er trængsel på infrastrukturen, kan selv små hændelser som fx en mindre ulykke ændre rejsetiden markant. Uforudsigeligheden betyder, at rejsende – for eksempel en forretningsrejsende der skal nå et møde – tager tidligere af sted for at være sikker på at være der til tiden. Ikke mindst for erhvervs- og pendlingstrafikken er der forment- lig en betydelig negativ effekt af trængsel på produktiviteten. I det omfang bedre infrastruktur reducerer trængslen, bliver det nemmere for virksomhederne og deres medarbejdere at for- udsige og planlægge transporttider og dermed udnytte arbejdstiden effektivt.

\paragraph{Levihnson og Petrin}
Hvad nu hvis jeg skriver noge

Syntaksen for at citere er \cite[pp. 211ff.]{melo2009meta}. 

 % paragraph levihnson_og_petring (end)


