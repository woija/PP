\subsection{Indledning}
Agglomeration er et nyere begreb, der i litteraturen kredser sig om de bredere økonomiske effekter, der vokser frem, når individer og virksomheder rykkere tættere på hinanden. Men idéen er gammel.

Agglomeration har spillet en afgørende rolle i vores civilisations udvikling helt fra de første byer begyndte at tage form for over 7000 år siden. For da produktiviteten i landbruget blev så tilpas høj, at der var overskud til at nogle kunne brødføde sig uden selv at deltage i dyrkning af jorden, samlede folk sig byerne. Her kunne mennesker specialisere og udvikle sig og nyde godt af det diversificerede marked, hvor innovation og viden kunne dyrkes og spredes.

Historien har gentaget sig lige siden dengang. Folk flytter fortsat mod byerne, hvor de bliver en del af stadigt mere diversificeret og specialiseret arbejdsmarked, hvor netop uddannelse og specialisering er en kardinaldyd.

Agglomeration har været en drivende faktor for henholdsvis nogen andre personer og andre glæder
