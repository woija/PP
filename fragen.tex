\section{Spørgsmål til Asger}
\begin{itemize}
	\item Er der noget galt i at bruge en Cobb Douglas produktionsfunktion til at estimere en eksternalitet?
	\item Produktionsfunktion - translog vs. CF?
	\item Der er helt sikkert et positivt bias i vores model. Er det tilfredsstillende blot at skrive sig ud af det, eller skal der korrieres på en eller anden måde. Bemærk, at vi kun har tilgang til private byerhverv.
	\item Hvad skal vores policy være?
	\begin{itemize}
	 	\item[a.] Planloven: placering af virksomheder, bystørrelsen, 3D-byer ...
	 	\item[b.] Klyngepolitik. Hvilke og hvor store er effekterne af disse?
	 	\item[c.] Government-bashing...
	 	\item[d.] Hvor meget skal artiklen/resume være en opsummering af opgaven? Og hvor meget skal det være et forsøg på at få noget i avisen? De to kan være vanskelige helt at samstemme.
	 	\begin{itemize}
	 		\item Vores umiddelbare ide ligger i at sige, at regeringen måske er lidt i konflikt med sig selv, når den på den ene side prøver at fremelske klyngestrategier (agglomerationer) i et videns og forskningsøjemed, mens den sender videnstunge arbejdsplader ud i provinsen ved udflytning af statslige arbejdspladser.
	 	\end{itemize}
	 \end{itemize} 
\end{itemize}